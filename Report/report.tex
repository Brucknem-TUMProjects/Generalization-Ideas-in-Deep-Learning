\documentclass[a4paper,10pt]{article}
\usepackage[utf8]{inputenc}
\usepackage[margin=3cm]{geometry}
\usepackage{amsmath}
\usepackage{amssymb}
\usepackage{amsthm}
\usepackage{fancyhdr}
\usepackage{seminar}
\usepackage{graphicx}
\usepackage{subfigure}
\usepackage{float}
\usepackage{hyperref}

\pagestyle{fancy}


%You can add theorem-like environments (e.g. remark, definition, ...) if you want
\newtheorem{theorem}{Theorem}

\title{Generalization Ideas in Deep Learning} % Replace with your title
\author{Marcel Bruckner} % Replace with your name
\institute{\textit{Seminar: Optimization and Generalization in Deep Learning}}

\makeatletter
\let\runauthor\@author
\let\runtitle\@title
\makeatother
\lhead{\runauthor}
\rhead{\runtitle}




\begin{document}

\maketitle

\begin{abstract}
Trying to understand the generalization abilities of deep neural networks several capacity measures are experimentally explored. Following the thoughts of \cite{neyshabur2017exploring} different norm-based capacity measures over the network weights and the sharpness as the robustness to pertubations on the parameter space are investigated. The measures are used in different experiments and the results are plotted against each other.
\end{abstract}

\section{What is generalization and why do we want it}

\section{What is a matrix norm and which do we use}
\subsection{What is a matrix norm}
\subsection{Why do we use matrix norms as measures for capacity bounds}
\subsection{Which norms do we use}
\begin{itemize}
    \item $l2$ norm
    \item $l1$-path norm
    \item $l2$-path norm
    \item spectral norm
    \item spectral norm
    \item spectral norm
    \item spectral norm
    \item spectral norm
\end{itemize}

\section{What is sharpness}
\newpage

\section{A few remarks}
Each report should include an introduction describing the problem, the motivations and a brief outline. The main approach should then be described and discussed in separate sections, followed by experimental results (when applicable) and conclusions.

\begin{itemize}
\item Please use citations when appropriate. Again, you are not expected to read through all the references appearing in your assigned paper. Add your citations in bibtex format into the file egbib.bib. An example is \cite{neyshabur2017exploring}.
\item You can use the theorem environment to write theorems. An example:
\begin{theorem}
\label{mytheorem}
Let $p$ be a prime number. Then, for any $a \in \mathbb{N}$, $a^p - a$ is evenly divisible by $p$. More formally,
\begin{equation}
a^p \equiv a \, (\mathrm{mod}\, p)\,.
\end{equation}
\end{theorem}
\item Please keep all your formulas numbered.
\item The report should be 4 to 6 pages long (not including citations).
\item Reports must be in English.
\item Please do not change the layout ({\em e.g.}, do not change page margins, font size, etc.).
\end{itemize}

\nocite{*}

\bibliographystyle{plain}
\bibliography{egbib}



\end{document}
